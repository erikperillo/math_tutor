\documentclass[12pt]{article}
\usepackage{graphicx}
\usepackage{float}
\usepackage{amsmath}
\usepackage[brazilian]{babel}
\usepackage[utf8]{inputenc}
\usepackage[T1]{fontenc}

\begin{document}

\title{Equações: Exercícios complementares}
\date{}
\maketitle

\newpage

\section{Exercícios}
\begin{enumerate}
	\item Resolva as equações a seguir para a incógnita:
	\begin{enumerate}
		\item $$\frac{1}{\pi*x}+\frac{1}{x} = \pi$$
		\item $$x^2 - 36 = 0$$
		\item $$2*\sqrt{x} + 3 = 4$$
		\item $$9 + \frac{2}{3} = 5x$$
		\item $$\frac{1}{x} + 3 = x + \frac{1}{x}$$
		\item $$x^3 + 3x^2 = 0$$
	\end{enumerate}

	\item Simplifique as equações a seguir:
	\begin{enumerate}
		\item $$3x + 9 = 15$$
		\item $$\frac{1}{x} + 14 = x + \frac{1}{x}$$
		\item $$0 = 81x + 90$$
		\item $$-2 -3x = 7x - 12x^2$$
		\item $$34 + 2x = 80 + 6x$$
	\end{enumerate}
\end{enumerate}

\newpage

\section{Respostas aos exercícios}
\begin{enumerate}
	\item
	\begin{enumerate}
		\item $$x = \frac{\pi + 1}{{\pi}^2}$$
		\item $$x = \pm 6$$
		\item $$x = {\left(\frac{1}{2}\right)}^2 = \frac{1}{4}$$
		\item $$x = \frac{29}{15}$$
		\item $$x = 3$$
		\item dividindo os dois lados por $x^2$: $$x + 3 = 0 \implies x = -3$$
	\end{enumerate}	

	\item
	\begin{enumerate}
		\item $$x + 3 = 5$$
		\item $$14 = x$$
		\item $$0 = 9x + 10$$
		\item $$2 + 3x = -7x + 12x \implies 2 = x(-7 + 12 - 3) = 2x$$
	\end{enumerate}
\end{enumerate}

\end{document}
