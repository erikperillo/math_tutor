\documentclass[11pt]{article}
\usepackage{graphicx}
\usepackage{float}
\usepackage{amsmath}
\usepackage{amsfonts}
\usepackage[brazilian]{babel}
\usepackage[utf8]{inputenc}
\usepackage[T1]{fontenc}

\begin{document}

\title{Matemática Elementar: Inequações}
\author{Erik Perillo}
\date{}
\maketitle
\begin{abstract}
Nesta etapa, falaremos sobre inequações.
\end{abstract}

\newpage

\tableofcontents

\newpage

\section{Introdução}
\paragraph{}
Nesta seção nós vamos aprender o que podemos e o que não podemos fazer com
inequações matemáticas. 
\paragraph{}
O que são inequações? Ora, uma inequação nada mais é que uma expressão de
\emph{desigualdade}. Lembra de equações? elas eram uma expressão de igualdade,
agora falaremos sobre expressões de desigualdade.
Assim, quando eu digo que $a > b$, eu estou dizendo que $a$ é maior que $b$.
O que é mais comum de se ver em inequações são \emph{incógnitas} que temos que 
descobrir o valor. Por exemplo, em: 
$$3x + 34 > 89$$
A incógnita é o $x$, ou seja, é o valor que queremos descobrir qual é. 

\section{Símbolos}
\paragraph{}
Os símbolos da inequação são muito simples. São estes:
\begin{itemize}
	\item $>$ -- maior que. Exemplo:
		  $$4 > 3$$
		  (quatro é maior que três)
	\item $\geq$ -- maior ou igual que. Exemplo:
		  $$4 \geq 4$$
		  (quatro é maior que 4 \textbf{ou} é igual a 4. Nesse caso, é igual.)
	\item $<$ -- menor que. Exemplo:
		  $$3 < 5$$
		  (três é menor que cinco)
	\item $\leq$ -- menor ou igual que. Exemplo:
		  $$4 \leq 5$$
		  (quatro é menor ou igual que 5. Nesse caso, é menor.)
\end{itemize}

\section{Técnicas de solução de inequações}
\paragraph{}
Você sabe solucionar uma equação? Então você sabe resolver uma inequação! Quer
dizer, quase isso. A dica geral é que, para resolver uma inequação, você tem
que fingir que aquilo é uma equação. \textbf{Há exceções} que serão explicadas
a seguir, mas primeiro vamos fingir que ela é como uma equação. Assim, para
resolver a inequação:
$$x + 3 > 5$$
O que eu faço? Oras, apenas finja que o sinal de maior ($>$) é um sinal de 
igualdade ($=$):
$$x + 3 = 5$$
Agora você pode solucionar, não? Fazendo as contas, temos que $x = 2$. Agora é
só trocar o sinal de volta e pronto:
$$x > 2$$
Resolvemos nossa inequação!

\paragraph{}
Assim como na equação, na inequação você pode:
\begin{itemize}
	\item Adicionar/Subtrair dos dois lados
	\item Multiplicar/Dividir dos dois lados
\end{itemize}

Só existe \textbf{um} porém. Quando multiplicamos os dois lados por um número
negativo, temos que inverter o sinal da desigualdade. Calma que eu vou explicar.
Veja a expressão:
$$-3 > -5$$
Ela está correta, certo? O número $-3$ realmente é maior que o $-5$. Agora,
se multiplicarmos os dois lados por $-1$, temos:
$$3 > 5$$
Epa! Isso tá errado! E tá mesmo, sabemos que 3 não é maior que 5. O problema
é que não invertemos o sinal de desigualdade. Se, ao invés de $>$, usarmos
$<$, temos:
$$3 < 5$$
Agora está certo! Então fica a regra: \textbf{Quando multiplicamos os dois 
lados por um número negativo, invertemos o sinal da inequação}. Se era
$\leq$, por exemplo, vira $\geq$.

Essa é a única regra que você precisa saber a mais. De resto, você pode usar
as regras de resolução de equações!

\newpage

\section{Exercícios}
\begin{enumerate}
	\item Resolva as inequações a seguir para a incógnita e dê um exemplo de
		  um número \textbf{racional} que obedece à inequação. Exemplo:
		  $$4x > 16 \implies x > 4$$
		  Como exemplo, temos 5, que é maior que 4.
	\begin{enumerate}
		\item $5x > 25$
		\item $x + 34 \leq 18$
		\item $\frac{x}{2} > 4$
		\item $x + 3 < 5$
		\item $3 - 4x \geq 12$
		\item $3 + x > 12$
		\item $-x \leq -9+3$
		\item $2x -9 -4x > -7$
	\end{enumerate}

	\item Marque as alternativas a seguir como verdadeiro ou falso.
	\begin{enumerate}
		\item $\pi > 4$
		\item $x > \frac{1}{x}$ se $x > 1$
		\item $-x > x$ se $x > 0$
	\end{enumerate}
\end{enumerate}

\newpage

\section{Respostas aos Exercícios}
\begin{enumerate}
	\item
	\begin{enumerate}
		\item $x > 5$ (Ex. 6)
		\item $x \leq -16$ (Ex. -16)
		\item $x > 8$ (Ex. 20)
		\item $x < 2$ (Ex. -21)
		\item $-x\geq\frac{9}{4}\implies x\leq-\frac{9}{4}$ (Ex. $-\frac{8}{4}$)
		\item $x > 9$ (Ex. $1001.34$)
		\item $x \geq 6$ (Ex. $6.1$)
		\item $x < 1$ (Ex. $0.3$)
	\end{enumerate}

	\item
	\begin{enumerate}
		\item Falso. $\pi = 3.141593$, mais ou menos, então é menor que $4$.
		\item Verdadeiro. Veja, por exemplo, 5. $\frac{1}{5}$ é $0.2$, o que é
			  menor que 5.
		\item Falso. Veja, por exemplo, 1. $-1$ não é maior que 1.
	\end{enumerate}
\end{enumerate}



\end{document}
