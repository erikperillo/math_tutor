\documentclass[11pt]{article}
\usepackage{graphicx}
\usepackage{float}
\usepackage{amsmath}
\usepackage{amsfonts}
\usepackage[brazilian]{babel}
\usepackage[utf8]{inputenc}
\usepackage[T1]{fontenc}

\begin{document}

\title{Matemática Elementar: Equações}
\author{Erik Perillo}
\date{}
\maketitle
\begin{abstract}
Nesta etapa, falaremos sobre uma das coisas mais importantes na matemática.
\end{abstract}

\newpage

\tableofcontents

\newpage

\section{Introdução}
\paragraph{}
Nesta seção nós vamos aprender o que podemos e o que não podemos fazer com
equações matemáticas. 
\paragraph{}
O que são equações? Ora, uma equação nada mais é que uma expressão de
\emph{igualdade}. Assim, quando eu digo que $a = b$, eu estou dizendo que $a$
e $b$ são \emph{idênticos}, eles equivalem um ao outro, \textbf{não importa o
que é a e o que é b}. O que é mais comum de se ver em equações são
\emph{incógnitas} que temos que descobrir o valor. Por exemplo, em:
$$3x + 34 = 89$$
A incógnita é o $x$, ou seja, é o valor que queremos descobrir qual é. 

\section{Técnicas de solução de equações}
\paragraph{}
Nesta seção, vamos ver modos comuns de solucionar uma equação.

\subsection{Adicionando e subtraindo dos dois lados}
\paragraph{}
Quando temos uma equação do tipo:
$$x + 3 = 4$$
O que podemos fazer para resolvê-la? Ora, é só passar o $3$ para o outro lado,
não? Ele fica negativo e obtemos $x = 4 - 3 = 1$. Por que podemos passar o $3$
para o outro lado? Isso porque, no fundo, estamos tirando o $3$ dos dois lados.
\paragraph{}
Por que podemos fazer isso? Pense da seguinte maneira: a gente não viu que os
dois lados da equação são exatamente iguais? Ou seja: $a = b$. Se um lado é
exatamente igual ao outro, eles são \textbf{a mesma coisa}. Então, se eu
adicionar/multiplicar/subtrair/dividir ou qualquer outra coisa em um lado e 
fizer o mesmo do outro, \textbf{eles vão continuar sendo a mesma coisa}.
\paragraph{}
Olhando a equação com outros olhos agora, temos:
$$x + 3 = 4$$
$$x + 3 - 3 = 4 - 3$$
$$x = 1$$
Note que temos que ter sagacidade. Eu não tirei justo o número $3$ à toa, mas
sim porque eu sabia que era o que me possibilitaria de ter o resultado. Outro
exemplo:
$$-7 + x = -10$$
$$-7 + x + 7 = -10 + 7$$
$$x = 7 - 10 = -3$$

\subsection{Multiplicando e dividindo dos dois lados}
\paragraph{}
Como foi dito na seção anterior, \emph{qualquer} coisa que façamos na equação,
contanto que seja dos dois lados, mantém a veracidade dela intacta. Vamos
agora ver o exemplo:
$$3x = 9$$
Pensando em passar o $3$ para o outro lado dividindo, obtemos $x = 3$. O que
fizemos, na verdade, foi a divisão certa:
$$3x = 9$$
$$\frac{(3x)}{3} = \frac{9}{3}$$
$$x = 3$$
Podemos fazer também multiplicações sábias:
$$\frac{x}{4} = -2$$
$$\left(\frac{x}{4}\right)*4 = (-2)*4$$
$$x = -8$$
\paragraph{}
Podemos também inverter o sinal das coisas. Na equação:
$$-x = -4$$
Podemos multiplicar os dois lados por $-1$ para obter:
$$-x*(-1) = -4*(-1)$$
$$x = 4$$
\paragraph{}
Temos que ter um pouco de cuidado! Pode-se dividir os dois lados por zero? Não,
isso é proibido. Se pudéssemos fazer isso, chegaríamos a resultados bem 
absurdos como, por exemplo, que $2 = 1$.

\subsection{Exponenciando dos dois lados}
\paragraph{}
Continuando com as operações, podemos agora elevar os dois lados a uma mesma
potência pra conseguir o nosso resultado. Por exemplo:
$$\sqrt{x} = 12$$
Elevando os dois lados ao quadrado, temos:
$${(\sqrt{x})}^2 = 12^2$$
$$x = 144$$
Podemos também tirar a raiz (o que é só elevar a $1/2$) dos dois lados:
$$x^2 = 16$$
$$\sqrt{x^2} = \sqrt{16}$$
$$x = \pm4$$
Por que colocamos $\pm$ no $4$ acima? Bom, tanto o $4$ quanto o $-4$, quando
elevados ao quadrado, podem dar $16$. Temos, então, que colocar o $\pm$ pois
qualquer um dos dois satisfaz a equação.
\paragraph{}
Temos, porém, que ter \textbf{cuidado}! Quando um dos lados é negativo, a gente
não pode fazer tirar raiz. Se fosse, por exemplo, $x^2 = -5$, não poderíamos 
tirar a raiz dos dois lados. Mais pra frente vamos ver que, fazendo certas 
coisas, vamos poder tirar a raiz, mas por enquanto assuma que é proibido.

\subsection{Misturando várias coisas}
\paragraph{}
No dia a dia você não vai encontrar uma equação que só precisa de uma operação
pra ser solucionada como a que vimos até agora. É comum vermos coisa do tipo:
$$3x - 24 = 45$$
O que fazemos, então? O jeito é irmos fazendo operações, \textbf{sempre nos
dois lados}, até conseguirmos \emph{isolar} o $x$. É sempre melhor começar com
adições/subtrações e isolar o $x$ de um lado:
$$3x - 24 = 45$$
$$3x - 24 + 24 = 45 + 24$$
$$3x = 69$$
Agora que temos só $x$ de um lado, podemos nos perguntar o que precisamos fazer
para que o $3$ suma. A resposta é dividir, claro!
$$\frac{3x}{3} = \frac{69}{3}$$
$$3 = 23$$
\paragraph{}
Se tivéssemos uma exponenciação ou raiz, é bom sempre primeiro subtrair e
multiplicar pra depois exponenciar:
$$4\sqrt{x} - 13 = 3$$
$$4\sqrt{x}- 13 + 13 = 3 + 13$$
$$4\sqrt{x} = 16$$
$$\frac{4\sqrt{x}}{4} = \frac{16}{4}$$
$$\sqrt{x}= 4$$
$${(\sqrt{x})}^2 = 4^2$$
$$x = 16$$

\newpage

\section{Exercícios}
\begin{enumerate}
	\item Resolva as equações a seguir para a incógnita:
	\begin{enumerate}
		\item $4x = 16$
		\item $x - 34 = 18$
		\item $\frac{x}{2} = 4$
		\item $x + 3 = 5$
		\item $3 + 4x = 12$
		\item $2x -9 -4x = -7$
		\item $3 + 2x = 3x + 8$
		\item $-x = -9+3$
		\item $3 + 3x + 10 -\frac{x}{2} = 34$
		\item $3x + 9 = 0$
		\item $34 + \frac{x}{3} = 4$
		\item $4 + \frac{1}{x} = 5 - \frac{3}{x}$
		\item $23 - 4y = 43$
		\item $4 - \sqrt{x} = 0$
		\item $3 + x^2 = 12$
	\end{enumerate}
\end{enumerate}

\newpage

\section{Respostas aos Exercícios}
\begin{enumerate}
	\item
	\begin{enumerate}
		\item $x = 4$
		\item $x = 52$
		\item $x = 8$
		\item $x = 2$
		\item $x = \frac{9}{4}$
		\item $x(2 - 4) = -7 + 9 \implies -2x = 2 \implies x = -1$
		\item Tire o $2x$ dos dois lados. $3 + 2x - 2x = 3x + 8 -2x \implies
			  x + 8 = 3 \implies x = -5$
		\item $x = 6$
		\item $\frac{6x - x}{2} = 21 \implies x = \frac{42}{5}$
		\item $x = -3$
		\item $x = -90$
		\item $\frac{4}{x} = 1 \implies x = 4$
		\item $y = -5$
		\item $x = 16$
		\item $x = \pm 3$
	\end{enumerate}
\end{enumerate}



\end{document}
