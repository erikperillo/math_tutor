\documentclass{article}
\usepackage{graphicx}
\usepackage{float}
\usepackage{amsmath}
\usepackage[brazilian]{babel}
\usepackage[utf8]{inputenc}
\usepackage[T1]{fontenc}

\begin{document}

\title{Matemática Elementar: Manipulações Algébricas}
\author{Erik Perillo}
\maketitle
\begin{abstract}
Nesta etapa, nós vamos ver os macetes mais usados no dia a dia das 
manipulações algébricas, mais conhecidas como continhas.
\end{abstract}

\newpage

\tableofcontents

\newpage

\section{Introdução}
\paragraph{}
Manipulações algébricas são importantíssimas para qualquer área da matemática
que envolva contas. Muitas vezes, as contas que fazemos se encontram em um
estado complicado, feio, difícil de se resolver. Com truques clássicos de
manipulação, é possível simplificar e resolver um montão de contas que você
vai encontrar por aí. Veja, por exemplo, a conta:
$$\frac{6x^2 + 36x + 48}{\frac{2}{1/3}}$$
Parece meio complicada, não? Entretanto, no fim desta etapa, você vai estar
capaz de bater o olho em algo como isso e perceber que ela é equivalente a:
$${(x + 3)}^2 - 1$$
Muito mais simples, não? Por isso é tão importante estar fera nessa parte!

\section{Operações e suas técnicas}
\paragraph{}
Nesta seção vamos ver as técnicas associadas a cada tipo de operação, uma de
cada vez.

\subsection{Adição e subtração}
\paragraph{}
\begin{itemize}
	\item $a - (-b) = a + b$
	\paragraph{}
	Exemplo:
	$$4 - (-3) = 4 + 3 = 7$$
	\item $b - a = -(a - b)$
	\paragraph{}
	Apenas distribua o -1 ($-(a - b) = -1*(a - b)$!!) para ver:
	$$-1*(a - b) = (-1*(a)) - (-1*(b)) = -a + b = b - a$$
	\paragraph{}
	Exemplos:
	\begin{enumerate}
		\item 
		$$-(5 - 4) = -(1) = -1$$
		$$(4 - 5) = -1$$
		\item 
		$$5 - (4 - 5) = 5 - (5 - 4) = 4$$
	\end{enumerate}
\end{itemize}

\subsection{Multiplicação}
\begin{itemize}
	\item $a*(b+c) = a*b + a*c$ (distributiva)
	\paragraph{}
	Exemplo:
	$$4*(5 + 6) = 4*(11) = 44$$
	$$4*5 + 4*6 = 20 + 24 = 44$$
	\item $a*b + a*c = a*(b + c)$ (distributiva!)
	\paragraph{}
	Note que essa é só a distributiva sendo feita ao contrário. Exemplo:
	$$6 + 9 = 15$$
	$$6 + 9 = 3*2 + 3*3 = 3*(2 + 3) = 3*(5) = 15$$
	\item $(a + b)*(c + d) = a*c + a*d + b*c + b*d$
	\paragraph{}
	Exemplo:
	$$(2+3)*(4+5) = (5)*(9) = 45$$
	$$(2+3)*(4+5) = 2*4 + 2*5 + 3*4 + 3*5 = 8 + 10 + 12 + 15 = 45$$
	Isso traz uma propriedade importante:
	$${(a+b)}^2 = a^2 + 2*a*b + b^2$$
	Para ver isso, é só fazer $(a+b)*(a+b)$ e você vai ter o resultado. Saber 
	isso de cabeça vai te poupar tempo, pois isso acontece bastante. Outra 
	forma é:
	$${(a-b)}^2 = a^2 - 2*a*b + b^2$$
\end{itemize}

\subsection{Divisão}
Antes de entrar nesta seção, vamos relembrar um conceito importante: 
\textit{frações}. O que são frações? \textbf{Uma fração é a representação de
uma divisão}. Quando você divide $10$ por $4$, por exemplo, você pode 
representar o resultado ($2.5$) através de uma fração:
$$\frac{10}{4}$$
A parte de cima, chamamos de \textit{numerador}. A parte de baixo, chamamos de
\textit{denominador}. Vamos agora ver as propriedades de frações.
\begin{itemize}
	\item \textbf{Dividir por um número é igual a multiplicar pelo inverso}.
	Veja a fração $\frac{4}{2}$, por exemplo. O número resultante é exatamente
	igual ao produto de $4$ com o inverso de $2$, ou seja,
	$4*(\frac{1}{2}) = 4*(0.5) = 2$. Saber disso se torna mais útil em
	casos como:
	$$\frac{4}{\frac{1}{3}}$$
	Como resulver isso? Oras, sabendo que dividir é multiplicar pelo inverso,
	podemos ver que:
	$$\frac{4}{\frac{1}{3}} = 4*3 = 12$$
	\item \textbf{Soma de frações}:
	$$\frac{a}{c} + \frac{b}{c} = \frac{(a + b)}{c}$$
	\paragraph{}
	Exemplo:
	$$\frac{10}{2} + \frac{4}{2} = 5 + 2 = 7$$
	$$\frac{10}{2} + \frac{4}{2} = \frac{(10 + 4)}{2} = \frac{14}{2} = 7$$
	Note que \textbf{o denominador tem que ser o mesmo entre as duas frações}.
	Não podemos, por exemplo, somar as partes de cima de $\frac{4}{3} + 
	\frac{5}{8}$, pois os denominadores são diferentes ($3$ e $8$). Mais pra 
	frente vamos ver como contornar isso. Por ora, podemos ver o motivo. Lembra
	que dividir por um número é multiplicar pelo inverso dele? Pois bem, podemos
	então escrever a expressão anterior na forma:
	$$\frac{10}{2} + \frac{4}{2} = 10*(\frac{1}{2}) + 4*(\frac{1}{2}) =$$
	$$(10 + 4)*(\frac{1}{2}) = \frac{(14)}{2} = 7$$
	Ou seja: no fundo, estamos aplicando a distributiva! então não faz sentido
	quando o denominador é diferente, pois então não podemos distribuir.
	\item \textbf{Multiplicação de frações}:
	$$(\frac{a}{b})*(\frac{c}{d}) = \frac{(a*c)}{(b*d)}$$
	Ou seja, multiplicamos os de cima com os de cima e os de baixo com os de
	baixo. Exemplo:
	$$\frac{4}{2} * \frac{9}{3} = 2 * 3 = 6$$
	$$\frac{4}{2} * \frac{9}{3} = \frac{(4*9)}{(2*3)} = \frac{36}{6} = 6$$
	\item \textbf{Simplificações de frações}:
	\paragraph{}
	Observe a fração:
	$$\frac{400}{300}$$
	Será que temos mesmo que dividir $400$ por $300$? Não há uma forma de 
	simplificar as coisas? Claro que há, amiguinho (a)! Sabemos que 
	$400 = 4*100$ e $300 = 3*100$, então podemos escrever:
	$$\frac{400}{300} = \frac{(4*100)}{(3*100)} = 
	(\frac{4}{3})*(\frac{100}{100}) = \frac{4}{3}*1 = \frac{4}{3}$$
	A gente aplicou a regra da multiplicação de frações pra separar as duas
	frações. Então fica a regra: \textbf{Quando o numerador e o denominador
	são múltiplos de um mesmo número, podemos dividir os dois por esse número}.
	No exemplo de cima, nós vimos que tanto $400$ quanto $300$ são 
	\textbf{múltiplos de 100}. A gente, então, pode \textbf{simplificar} 
	a fração, ou seja, sumir com o $100$ dividindo o $400$ e o $300$ por $100$.
	\paragraph{}
	Outros exemplos:
	\begin{enumerate}
		\item $\frac{8}{6} = \frac{2*4}{2*3} = \frac{4}{3}$
		\item $\frac{169}{26} = \frac{13*13}{13*2} = \frac{13}{2}$
		\item $\frac{1000}{100} = \frac{100*10}{100*1} = \frac{10}{1} = 10$
	\end{enumerate}
	\item \textbf{Adição de frações com denominadores diferentes}:
	\paragraph{}
	Veja a expressão:
	$$\frac{4}{3} + \frac{7}{5}$$
	Como adicionamos isso? Podemos reescrever a expressão como:
	$$\frac{4}{3}*\frac{5}{5} + \frac{7}{5}*\frac{3}{3}$$
	Veja que o número resultante não mudou em nada: a gente apenas multiplicou
	por $5/5$ e por $3/3$, ou seja, multiplicamos por $1$. Então o resultado
	tem que ser a mesma coisa! mas olha agora o que dá pra fazer:
	$$\frac{4}{3}*\frac{5}{5} + \frac{7}{5}*\frac{3}{3} = \frac{4*5}{3*5} + 
	\frac{7*3}{5*3}$$
	Opa! Agora os denominadores das duas frações são iguais! Agora sim podemos
	adicionar as duas:
	$$\frac{4*5}{3*5} + \frac{7*3}{5*3} = \frac{4*5}{15} + \frac{7*3}{15} =
	\frac{(4*5 + 7*3)}{15} = \frac{41}{15}$$
	Mágico, não? Como regra geral, a gente tem:
	$$\frac{a}{b} + \frac{c}{d} = \frac{(a*d + b*c)}{b*d}$$
\end{itemize}

\subsection{Exponenciação}
Primeiramente, vamos dar nomes aos bois. Quando vemos algo como:
$$4^5$$
Nós dizemos que o $4$ é a \textit{base} e o $5$ é o \textit{expoente}.
\begin{itemize}
	\item $a^0 = 1$
	\paragraph{}
	Isso mesmo. \textbf{Para qualquer número} (tirando o zero), elevar ele
	a zero dá $1$. Exemplos:
	\begin{enumerate}
		\item $1^0 = 1$
		\item $1343434^0 = 1$
		\item $0.34398^0 = 1$
		\item ${-10}^0 = 1$
	\end{enumerate}
	\item ${(a*b)}^c = a^c*b^c$
	\paragraph{}
	Exemplo:
	$${(3*2)}^2 = {(6)}^2 = 36$$
	$${(3*2)}^2 = 3^2*2^2 = 9*4 = 36$$
	Note que isso não é verdade para a soma:
	$${(3+2)}^2 = {(5)}^2 = 25$$
	O que é diferente de:
	$$3^2 + 2^2 = 9 + 4 = 13$$
	\item $a^b*a^c = a^{(b + c)}$
	\paragraph{}
	Exemplo:
	$$3^2*3^5 = (9)*(243) = 2187$$	
	$$3^2*3^5 = 3^{(2 + 5)} = 3^{(7)} = 2187$$	
	Note que \textbf{a base tem que ser a mesma}.
	\item $\frac{a^b}{a^c} = a^{(b - c)}$
	\paragraph{}
	Exemplos:
	\begin{enumerate}
		\item 
		$$\frac{3^5}{3^2} = \frac{243}{9} = 27$$	
		$$\frac{3^5}{3^2} = 3^{(5-2)} = 3^3 = 27$$	
		\item
		$$\frac{3^2}{3^5} = \frac{9}{243} = 0.037$$	
		$$\frac{3^2}{3^5} = 3^{(2-5)} = 3^{-3} = 0.037$$	
	\end{enumerate}
	Note que \textbf{a base tem que ser a mesma}.
	\item ${(a^b)}^c = a^{(b*c)}$
	\paragraph{}
	Exemplo:
	$${(3^2)}^3 = {(9)}^3 = 729$$
	$${(3^2)}^3 = 3^{(2*3)} = 3^6 = 729$$
\end{itemize}

\subsection{Radiciação}
Primeiro, vamos notar uma coisa: 
$$\sqrt{4} = 4^{1/2} = 2$$
Essa é uma propriedade importantíssima da radiciação! Veja o caso geral:
$$\sqrt[n]{a} = a^{\frac{1}{a}}$$
Ou seja, tirar a raiz $n$ de um número é a mesma coisa que elevá-lo a $1/n$!
Uau! Isso é especialmente importante porque, se olharmos desse jeito, vamos
ver que as mesmas regras da exponenciação se aplicam à radiciação. Vamos 
listá-las:
\begin{itemize}
	\item $\sqrt[n]{a*b} = \sqrt[n]{a}*\sqrt[n]{b}$
	\paragraph{}
	Exemplos:
	\begin{enumerate}
		\item 
		$$\sqrt{4*9} = \sqrt{36} = 6$$
		$$\sqrt{4*9} = \sqrt{4}*\sqrt{9} = 2*3 = 6$$
		\item
		$$\sqrt{5}*\sqrt{3} = \sqrt{5*3} = \sqrt{15}$$
	\end{enumerate}
	\item $\sqrt[n]{a}*\sqrt[m]{a} = \sqrt[\frac{n*m}{n+m}]{a}$
	\paragraph{}
	Isso vem de:
	$$\sqrt[n]{a}*\sqrt[m]{a} = a^{\frac{1}{n}}*a^{\frac{1}{m}} = $$
	$$a^{\frac{1}{n} + \frac{1}{m}} = a^{\frac{m+n}{m*n}}$$
\end{itemize}

\paragraph{}
Parabéns! Agora você já sabe todos os macetinhos básicos que vão te permitir
fazer uma infinidade de contas.

\section{Exercícios}
\begin{enumerate}
	\item Resolva as expressões de soma e subtração:
	\begin{enumerate}
		\item $4 - (-3)$
		\item $5 - (34 - (43 - 100))$ 
	\end{enumerate}

	\item Resolva as expressões de multiplicação:
	\begin{enumerate}
		\item $4 * (-3 -23)$
		\item $(1 + 2)*(3-4)$
	\end{enumerate}
	
	\item Aplique a distributiva na expressão:
	$$(a + b*c)*(c - d)$$

	\item Escreva as expressões a seguir como uma soma/subtração de quadrados.
	Exemplo:
	$$4 + 12 + 9 \implies 2^2 + 2*2*3 + 3^2 = {(2+3)}^2$$
	\begin{enumerate}
		\item $25 + 40 + 16$
		\item $25 - 30 + 9$
		\item $16 + 32 + 16$
	\end{enumerate}

	\item Resolva as expressões com fração. 
	Deixe o resultado em forma de fração. Exemplo:
	$$\frac{1}{3} + \frac{6}{3} = \frac{7}{3}$$
	\begin{enumerate}
		\item $\frac{3}{2} + \frac{5}{2}$	
		\item $\frac{1}{293} + \frac{22}{293} - \frac{14}{293}$
		\item $\frac{2}{3} * \frac{4}{5}$
		\item $\frac{\frac{10}{5}}{\frac{3}{7}}$
		\item $(\frac{(\frac{10}{5})}{(\frac{3}{5})}) * 
		\frac{2}{4} - \frac{3}{12}$
		\item $\frac{3}{5} + \frac{5}{6}$
		\item $\frac{2}{3} - \frac{7}{6}$
	\end{enumerate}
\end{enumerate}


\end{document}
