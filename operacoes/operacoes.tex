\documentclass{article}
\usepackage{graphicx}
\usepackage{float}
\usepackage[brazilian]{babel}
\usepackage[utf8]{inputenc}
\usepackage[T1]{fontenc}

\begin{document}

\title{Matemática Elementar: Operações}
\author{Erik Perillo}
\maketitle

\begin{abstract}
Nesta aula, vamos tratar das operações mais básicas que se pode fazer em 
problemas de matemática. Estar fera nessas coisas é muito importante pra 
que você possa prosseguir com segurança na construção do seu conhecimento em
matemática até chegar em coisas que você verá na faculdade, como cálculo e 
geometria analítica.
\end{abstract}

\newpage

\tableofcontents

\newpage

\section{Introdução}
\paragraph{}
O objetivo desta etapa é fazer com que você se lembre das regras mais básicas
usadas nas continhas de matemática, como adição, multiplicação etc. 
Pela importância dessa parte para todo o resto, nós vamos começar 
desde o básico do básico (a ponto de parecer bobo), mas é melhor repetirmos 
o que já sabemos e ter certeza de que sabemos do que ficar na dúvida eterna.
Vamos começar a rever as operações matemáticas.
\paragraph{}
O que, exatamente, são operações?
A primeira coisa relacionada a matemática que toda criança (normal) aprende a
fazer é a famosa conta de mais e menos, ou \textit{adição} e \textit{subtração}, 
representadas pelos símbolos $+$ e $-$. Logo depois, aprendemos a 
\textit{multiplicação} e a \textit{divisão}, representados por $*$ e $/$. Todas
essas coisinhas, $+ - * /$, são \textit{operações}. 
\paragraph{}
\textbf{Uma operação é algo que se aplica a dois números para se obter um novo
número}. A adição, por exemplo, pega dois números e, com eles, forma outro 
número que é a soma deles. Por exemplo: $2 + 3$. Sabemos que o resultado é $5$.
Veja $2*4$. Sabemos que o resultado é $8$. Já dá pra perceber como a soma e a 
multiplicação são mesmo operações. No caso da adição, \textbf{pegamos dois 
números} ($2$ e $3$) e \textbf{obtivemos um novo número}, no caso, $5$. 
\paragraph{}
\textbf{Toda operação tem um símbolo que a representa}. 
No caso da adição, sabemos que é $+$.
\paragraph{}
Como sabemos o que fazer quando vemos o símbolo de $+$? Quer dizer: como 
sabemos que temos que somar os números? Sabemos disso porque sabemos como a
adição funciona, ou seja, sabemos quais as regras da adição. 
\textbf{Toda operação tem regras de como ela deve funcionar}.
\paragraph{}
Nesta etapa, vamos aprender mais sobre os nomes formais das regrinhas/termos que 
usamos em todo tipo de operação matemática, e vamos ver as regras das operações
que mais usamos no dia a dia: \textit{adição, subtração, multiplicação, divisão,
e exponenciação}.

\section{Propriedades de Operações}
\paragraph{}
Nesta seção, vamos ver quais as características que cada operação pode ter.

\subsection{Precedência}
\paragraph{}
Observe a conta a seguir:
\begin{equation}
34 + 2*5
\end{equation}
Qual o resultado dessa conta? Sabemos que é $34 + 10 = 44$. E a conta 
a seguir: 
\begin{equation}
12 - 2*(3 - 1) + 4
\end{equation}
Quanto vale? Deve ser fácil de ver que dá $12 - 2*2 + 4 = 12$. Agora volte e 
olhe para as duas contas. Como nós conseguimos resolvê-las? Como sabemos que, em
$34 + 2*5$, nós temos que primeiro multiplicar o $2$ pelo $5$ e depois somar com
o $34$? Isso é porque sabemos que a multiplicação tem \textit{precedência} maior
que a adição, ou seja, \textbf{ela deve ser feita primeiro}. Na segunda conta,
sabemos que temos que fazer a conta no parênteses primeiro, mesmo que dentro
dele existe uma conta de subtração. Isso porque sabemos que \textbf{o parênteses
tem a maior precedência de todas}. Não importa a conta, sabemos que temos que 
primeiro resolver o que está no parênteses e depois ir para o resto. 
E se tivermos algo como:
$$3 + 12*(12 + 4*(4 - 2))$$
O que fazemos? Tem um parênteses dentro do outro! Oras, cara pálida, é fácil:
apenas aplique a mesma regra quando entrar no parênteses! Então, devemos olhar
para $3 + 12*(12 + 4*(4 - 2))$ e perguntar: \textbf{qual operação tem maior 
precedência?} então vemos que é o que está dentro do parênteses, que é 
$12 + 4*(4 - 2)$. Por que não fazer a mesma pergunta de novo? Se você perguntar
novamente qual operação tem maior precedência, vai ver que é o parênteses de 
novo. Assim, vai conseguir fazer $4 - 2 = 2$, e vai ter $12 + 4*4 = 28$. Então,
olhando passo a passo, você vai ter algo como:
$$3 + 12*(12 + 4*(4 - 2)) =$$
$$3 + 12*(12 + 4*(2)) =$$
$$3 + 12*(20) =$$
$$3 + 240 = 243$$
Lindo, não? Agora você já sabe tudo sobre \textit{precedência}!

\subsection{Comutatividade}
\paragraph{}
Vamos voltar para a expressão $34 + 2*5$ da seção anterior. Lembre-se que o 
valor que tivemos foi $44$. E se, agora, fizermos uma coisa um pouco diferente,
algo do tipo:
$$34 + 5*2$$
Caso você não tenha percebido o que mudou, o $5$ trocou de lugar com o $2$ na 
multiplicação. Que valor vamos ter agora? Fazendo as contas, temos 
$34 + 10 = 44$, que é o mesmo valor de antes! Isso sugere uma coisa: quando 
mudamos os números de ordem na multiplicação, \textbf{o resultado é o mesmo}.
$2*5 = 5*2 = 10$. Quando uma operação tem essa propriedade, dizemos que a 
operação é \textit{comutativa}. 
\paragraph{}
Nem toda operação é comutativa! Veja a 
subtração, por exemplo. Pegue $5 - 2$. Sabemos que é $3$. Pegue agora $2 - 5$.
O resultado não é mais $3$, mas sim $-3$! Isso mostra que $2 - 5$ não é igual a
$5 - 2$ e, então, a subtração não é comutativa!
E isso é tudo o que você tem que saber sobre essa propriedade.

\subsection{Associatividade}
\paragraph{}
Veja a expressão:
$$5 + 7 + 11$$
Fazendo a conta, temos $12 + 11 = 23$. Note que primeiro somamos $5$ a $7$ e 
só depois somamos o resultado, $12$, com o $11$. E se tivéssemos primeiro 
somado o $7$ com o $11$? Teríamos $5 + 18 = 23$. 
O resultado foi o mesmo, não importando a ordem com que fizemos as contas, 
ou seja:
$$5 + 7 + 11 =$$
$$(5 + 7) + 11 =$$
$$5 + (7 + 11) = 23$$
Notamos, então, que a adição é uma operação \textit{associativa}, ou seja:
\textbf{fazer a conta começando da esquerda pra direita ou da direita pra 
esquerda tanto faz}. Em qualquer operação que isso aconteça diz-se que ela é
\textit{associativa}.

\subsection{Distributividade}
\paragraph{}
Falar de distributividade só faz sentido se falarmos de duas operações. 
Não faz sentido, por exemplo, falar que a multiplicação tem distributividade.
Faz sentido, entretanto, falar que a multiplicação tem distributividade com
a adição. Viu a diferença? \textbf{Distributividade é uma relação entre duas
operações}.
O que raios é isso? Veja a expressão:
$$3*(4 + 5)$$
Qual o resultado? Sabemos que é $3*(9) = 27$. Agora veja a conta:
$$3*4 + 3*5$$
O que dá isso? Notamos que dá $12 + 15 = 27$. Hum, interessante: deu a mesma
coisa que a conta anterior, mesmo que elas pareçam diferentes. Isso sempre vai
ser verdade, porque a multiplicação é \textit{distributiva} com relação à 
adição, ou seja, Você pode multiplicar os números dentro do parênteses
e depois adicioná-los um a um. Veja que não precisam ser só dois números dentro
do parênteses, podem ser quantos você quiser:
$$2*(3 + 4 + 5 + 6) =$$
	$$(2*3) + (2*4) + (2*5) + (2*6)=$$
$$6 + 8 + 10 + 12 = 36$$
Se tivéssemos somado tudo primeiro, teríamos $2*(18)$, que também é igual a 
$36$. Então vemos que a multiplicação é distributiva com relação à adição porque
\textbf{podemos distribuir o $2$ para dentro do parênteses, multiplicando todo 
número por $2$, e depois somar todos os resultados das multiplicações}. 
\paragraph{}
Note que o contrário não precisa ser verdade! A adição, por exemplo, não é 
distributiva com relação à multiplicação. Veja a expressão:
$$2 + (4*5)$$
O resultado é $2 + 20 = 22$. Se distribuíssemos o $2$, entretanto, teríamos:
$$(2+4) * (2+5)$$
O que dá $6*7 = 42$, que obviamente não é igual a $22$.
\paragraph{}
Parabéns! Agora que você já sabe o que é distributividade, você já conhece 
todas as propriedades de operações!

\section{Operações comuns}
Nesta seção, vamos analisar as operações matemáticas mais famosas: 
\textit{adição, subtração, multiplicação, divisão, exponenciação e radiciação}.

\subsection{Adição}
\paragraph{}
A adição é a operação mais simples que se pode imaginar. Seu símbolo é $+$.
Todos sabemos o que fazer quando temos dois números como entrada para a operação
de adição: devemos somá-los. 
\paragraph{}
A adição é:
\begin{itemize}
\item Comutativa. Exemplo:
$$4 + 5 = 9$$
$$5 + 4 = 9$$
\item Associativa. Exemplo:
$$3 + 2 + 4 = 9$$
$$(3 + 2) + 4 = (5) + 4 = 9$$
$$3 + (2 + 4) = 3 + (6) = 9$$
\end{itemize}

\subsection{Subtração}
\paragraph{}
A subtração é o oposto da adição. Seu símbolo é $-$.
\paragraph{}
A subtração \textbf{não} é comutativa. Exemplo:
$$4 - 5 = -1$$
$$5 - 4 = 1$$
Ela também \textbf{não} é associativa. Veja:
$$(3 - 2) - 4 = 1 - 4 = -3$$
$$3 - (2 - 4) = 3 - (-2) = 3 + 2 = 5$$

\subsection{Multiplicação}
\paragraph{}
Seu símbolo é $*$.
A multiplicação pode ser vista como uma adição repetida. Isso mesmo! Ela só
diz quantas vezes você deve repetir uma soma. Veja, por exemplo:
$$4 + 4 + 4$$
O resultado é $12$. Vemos que o $4$ é somado três vezes. Se fizermos:
$$3*4$$
O resultado também é 12, o que mostra que a multiplicação é só a adição 
repetida!
\paragraph{}
Multiplicação tem precedência maior sobre adição/subtração. Então:
$$2 + 3*4 - 5 = 2 + (3*4) - 5 = 2 + (12) - 5 = 9$$
\paragraph{}
As multiplicação é:
\begin{itemize}
\item Comutativa. Exemplo:
$$4*5 = 20$$
$$5*4 = 20$$
\item Associatividade. Exemplo:
$$3*2*4 = 24$$
$$(3*2)*4 = (6)*4 = 24$$
$$3*(2*4) = 3*(8) = 24$$
\item Distributiva com relação à adição. Exemplo:
$$2*(3 + 4) = 2*(7) = 14$$
$$(2*3) + (2*4) = (6) + (8) = 14$$
\item Distributiva com relação à subtração. Exemplo:
$$2*(3 - 4) = 2*(-1) = -2$$
$$(2*3) - (2*4) = (6) - (8) = -2$$
\end{itemize}

\subsection{Divisão}
\paragraph{}
Seu símbolo é $/$.
Assim como a subtração é o oposto da adição, a divisão é o oposto da 
multiplicação.
\paragraph{}
Divisão tem precedência maior sobre adição/subtração. Então:
$$2 + 4/2 - 5 = 2 + (4/2) - 5 = 2 + (2) - 5 = -1$$
\paragraph{}
A divisão é:
\begin{itemize}
\item Distributiva com relação à adição. Exemplo:
$$(3 + 5)/2 = (8)/2 = 4$$
$$(3/2) + (5/2) = 1.5 + 2.5 = 4$$
\item Distributiva com relação à subtração. Exemplo:
$$(3 - 5)/2 = (-2)/2 = -1$$
$$(3/2) - (5/2) = 1.5 - 2.5 = -1$$
\end{itemize}
A divisão \textbf{não} é comutativa. Exemplo:
$$4 / 5 = 0.8$$
$$5 / 4 = 1.25$$
Ela também \textbf{não} é associativa. Veja:
$$(3 / 2) / 4 = (1.5)/4 = 0.375$$
$$3 / (2 / 4) = 3 / (0.5) = 6$$

\subsection{Exponenciação}
\paragraph{}
Assim como a multiplicação é a adição repetida, a exponenciação é a 
multiplicação repetida. Dá uma olhada:
$$4 * 4 * 4$$
Isso dá $64$. A multiplicação do 4 por ele mesmo foi feita três vezes. 
Agora olhe:
$$4^3$$
Isso também dá $64$, o que mostra que exponenciação é só a multiplicação
repetida!
\paragraph{}
Exponenciação tem precedência maior sobre multiplicação/divisão. Então:
$$2 * 4^2 - 5 = 2 * (4^2) - 5 = 2 * (16) - 5 = 36 - 5 = 31$$
\paragraph{}
A exponenciação é:
\begin{itemize}
\item Distributiva com relação à multiplicação. Exemplo:
$${(3*2)}^2 = {(6)}^2 = 36$$
$$(3^2)*(2^2) = (9)*(4) = 36$$
\item Distributiva com relação à divisão. Exemplo:
$${(4/2)}^2 = {(2)}^2 = 4$$
$$(4^2)/(2^2) = (16)/(4) = 4$$
\end{itemize}
Note que o contrário não é verdade, ou seja, nem a multiplicação nem a 
divisão são distributivas com relação à exponenciação! Veja:
$$4^{(2*3)} = 4^6 = 4096$$
$$(4^2)*(4^3) = 1024$$
As exponenciação \textbf{não} é comutativa. Exemplo:
$$2^3 = 8$$
$$3^2 = 9$$
Ela também \textbf{não} é associativa. Exemplo:
$${(2^3)}^4 = 8^4 = 4096$$
$2^{(3^4)} = 2^{81}$ é um número tão grande que nem vale a pena colocar aqui!

\subsection{Radiciação}
\paragraph{}
A radiciação é a operação inversa da exponenciação. 
Primeiro, veja como é uma radiciação do número $4$:
$$\sqrt{4}$$
Quanto dá isso? Bom, do jeito que está acima, nós estamos olhando para a 
\textit{raiz quadrada} de $4$. Tirar a raiz quadrada de um número quer dizer:
\textbf{Que número multiplicado por ele mesmo dá esse número na raiz?}. Veja
o exemplo do $4$: Que número multiplicado por ele mesmo dá $4$? Ora, é o 2,
veja: $2*2 = 4$. Assim, $\sqrt{4} = 2$.
Mas e se eu quiser saber, por exemplo, qual número multiplicado por ele 
\textbf{três} vezes dá $8$? Isso é diferente da raiz quadrada que vimos acima.
Queremos saber qual é o número $x$ tal que $x*x*x = 8$. Representamos esse
número por $\sqrt[3]{8}$. Viu que apereceu um número 3 ali em cima? Pois é. Na
raiz quadrada, esse número é $2$, mas a gente esconde o número $2$ quando é uma
raiz quadrada. Entretando, $\sqrt[2]{4} = \sqrt{4}$, é a mesma coisa.
\paragraph{}
Radiciação tem precedência maior sobre multiplicação/divisão. Então:
$$2 * \sqrt{4} - 5 = 2 * (\sqrt{4}) - 5 = 2 * (2) - 5 = 4 - 5 = -1$$
\paragraph{}
As radiciação \textbf{não} é comutativa. Exemplo:
$$\sqrt[3]{8} = 2$$
$$\sqrt[8]{3} = 1.147$$
A exponenciação é:
\begin{itemize}
\item Distributiva com relação à multiplicação. Exemplo:
$$\sqrt[2]{(4*4)} = \sqrt[2]{16} = 4$$
$$\sqrt[2]{(4)}*\sqrt[2]{(4)} = 2*2 = 4$$
\item Distributiva com relação à divisão. Exemplo:
$$\sqrt[2]{(16/4)} = \sqrt[2]{4} = 2$$
$$\sqrt[2]{(16)}/\sqrt[2]{(4)} = 4/2 = 4$$
\end{itemize}

\section{Exercícios}

\subsection{Seção 2}
\begin{enumerate}
	\item Ordene as precedências do parênteses, exponenciação, multiplicação e 
adição, da menor pra maior.
	\item Dê um exemplo de uma operação que é comutativa e uma que não é. Mostre
com números um exemplo para cada uma das duas.
	\item Dê um exemplo de uma operação que é associativa e uma que não é. 
Mostre com números um exemplo para cada uma das duas.
	\item Faz sentido dizer que a multiplicação é distributiva? Por quê?
\end{enumerate}

\subsection{Seção 3}
\begin{enumerate}
	\item Resolva as expressões:
	\begin{enumerate}
		\item $3 + 4*5$
		\item $(3 + 4)*5$
		\item $5*(3 + 4)$
		\item $11 - (3 + 4) + 5*(3 - (3 - 5))*2$
	\end{enumerate}
	\item Resolva as expressões (use uma calculadora para facilitar):
	\begin{enumerate}
		\item $3 * 4^3$
		\item ${(3 - 4)}^3$
		\item ${(3 * 4)}^3$
		\item $1 - (3^4) + 5*4^2*3^3$
	\end{enumerate}
	\item Suponha que uma pessoa não sabe qual a ordem de precedência das
operações. Ela só sabe que o parênteses tem que ser feito primeiro.
Ajude essa pessoa colocando os parênteses nos lugares certos para que ela
faça a operação corretamente. Por exemplo:
$$4 + 3*2 - 2*4^2$$
Tem que virar:
$$(4 + (3*2)) - (2*(4^2))$$
Sua vez! Complete com parênteses a seguinte expressão:
$$3*2 + 2*3^4*5$$
\end{enumerate}

\newpage

\section{Respostas aos exercícios}
\subsection{Seção 2}
\begin{enumerate}
	\item Adição, multiplicação, exponenciação, parênteses.
	\item 
	\begin{itemize}
		\item Comutativa: adição. Exemplo:
$$3 + 4 = 7$$
$$4 + 3 = 7$$
		\item Não comutativa: divisão. Exemplo:
$$4 / 2 = 2$$
$$2 / 4 = 0.5$$
	\end{itemize}
	\item 
	\begin{itemize}
		\item Associativa: multiplicação: Exemplo:
$$(2 * 3) * 4 = 6*4 = 24$$
$$2 * (3 * 4) = 2*12 = 24$$
		\item Não associativa: subtração: Exemplo:
$$(1 - 2) - 3 = (-1) - 3 = -4$$
$$1 - (2 - 3) = 1 - (-1) = 1 + 1 = 2$$
	\end{itemize}
	\item Não, pois distributividade só tem sentido quando se diz respeito a
duas operações.
\end{enumerate}

\subsection{Seção 3}
\begin{enumerate}
	\item 
	\begin{enumerate}
		\item $3 + 4*5 = 3 + 20 = 23$
		\item $(3 + 4)*5 = (7)*5 = 35$
		\item $5*(3 + 4) = 5*(7) = 35$
		\item $11 - (3 + 4) + 5*(3 - (3 - 5))*2 = 11 - (7) + 5*(3 - (-2))*2 =
11 - 7 + 5*(5)*2 = 4 + 25*2 = 54$
	\end{enumerate}
	\item 
	\begin{enumerate}
		\item $3 * 4^3 = 3*(64) = 192$
		\item ${(3 - 4)}^3 = {(-1)}^3 = -1$
		\item ${(3 * 4)}^3 = {(12)}^3 = 1728$
		\item $1 - (3^4) + 5*4^2*\sqrt[3]{3} = 1 - (81) + 5*(16)*(1.442) =
1 - 81 + 115.36 = 35.36$
	\end{enumerate}
	\item $3*2 + 2*3^4*5$ vira:
$$(3*2) + ((2*(3^4))*5)$$
\end{enumerate}


\end{document}
